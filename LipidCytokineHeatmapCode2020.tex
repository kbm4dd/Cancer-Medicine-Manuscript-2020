\documentclass[]{article}
\usepackage{lmodern}
\usepackage{amssymb,amsmath}
\usepackage{ifxetex,ifluatex}
\usepackage{fixltx2e} % provides \textsubscript
\ifnum 0\ifxetex 1\fi\ifluatex 1\fi=0 % if pdftex
  \usepackage[T1]{fontenc}
  \usepackage[utf8]{inputenc}
\else % if luatex or xelatex
  \ifxetex
    \usepackage{mathspec}
  \else
    \usepackage{fontspec}
  \fi
  \defaultfontfeatures{Ligatures=TeX,Scale=MatchLowercase}
\fi
% use upquote if available, for straight quotes in verbatim environments
\IfFileExists{upquote.sty}{\usepackage{upquote}}{}
% use microtype if available
\IfFileExists{microtype.sty}{%
\usepackage{microtype}
\UseMicrotypeSet[protrusion]{basicmath} % disable protrusion for tt fonts
}{}
\usepackage[margin=1in]{geometry}
\usepackage{hyperref}
\hypersetup{unicode=true,
            pdftitle={Heatmap Lipid Cytokines},
            pdfauthor={Katie},
            pdfborder={0 0 0},
            breaklinks=true}
\urlstyle{same}  % don't use monospace font for urls
\usepackage{color}
\usepackage{fancyvrb}
\newcommand{\VerbBar}{|}
\newcommand{\VERB}{\Verb[commandchars=\\\{\}]}
\DefineVerbatimEnvironment{Highlighting}{Verbatim}{commandchars=\\\{\}}
% Add ',fontsize=\small' for more characters per line
\usepackage{framed}
\definecolor{shadecolor}{RGB}{248,248,248}
\newenvironment{Shaded}{\begin{snugshade}}{\end{snugshade}}
\newcommand{\KeywordTok}[1]{\textcolor[rgb]{0.13,0.29,0.53}{\textbf{{#1}}}}
\newcommand{\DataTypeTok}[1]{\textcolor[rgb]{0.13,0.29,0.53}{{#1}}}
\newcommand{\DecValTok}[1]{\textcolor[rgb]{0.00,0.00,0.81}{{#1}}}
\newcommand{\BaseNTok}[1]{\textcolor[rgb]{0.00,0.00,0.81}{{#1}}}
\newcommand{\FloatTok}[1]{\textcolor[rgb]{0.00,0.00,0.81}{{#1}}}
\newcommand{\ConstantTok}[1]{\textcolor[rgb]{0.00,0.00,0.00}{{#1}}}
\newcommand{\CharTok}[1]{\textcolor[rgb]{0.31,0.60,0.02}{{#1}}}
\newcommand{\SpecialCharTok}[1]{\textcolor[rgb]{0.00,0.00,0.00}{{#1}}}
\newcommand{\StringTok}[1]{\textcolor[rgb]{0.31,0.60,0.02}{{#1}}}
\newcommand{\VerbatimStringTok}[1]{\textcolor[rgb]{0.31,0.60,0.02}{{#1}}}
\newcommand{\SpecialStringTok}[1]{\textcolor[rgb]{0.31,0.60,0.02}{{#1}}}
\newcommand{\ImportTok}[1]{{#1}}
\newcommand{\CommentTok}[1]{\textcolor[rgb]{0.56,0.35,0.01}{\textit{{#1}}}}
\newcommand{\DocumentationTok}[1]{\textcolor[rgb]{0.56,0.35,0.01}{\textbf{\textit{{#1}}}}}
\newcommand{\AnnotationTok}[1]{\textcolor[rgb]{0.56,0.35,0.01}{\textbf{\textit{{#1}}}}}
\newcommand{\CommentVarTok}[1]{\textcolor[rgb]{0.56,0.35,0.01}{\textbf{\textit{{#1}}}}}
\newcommand{\OtherTok}[1]{\textcolor[rgb]{0.56,0.35,0.01}{{#1}}}
\newcommand{\FunctionTok}[1]{\textcolor[rgb]{0.00,0.00,0.00}{{#1}}}
\newcommand{\VariableTok}[1]{\textcolor[rgb]{0.00,0.00,0.00}{{#1}}}
\newcommand{\ControlFlowTok}[1]{\textcolor[rgb]{0.13,0.29,0.53}{\textbf{{#1}}}}
\newcommand{\OperatorTok}[1]{\textcolor[rgb]{0.81,0.36,0.00}{\textbf{{#1}}}}
\newcommand{\BuiltInTok}[1]{{#1}}
\newcommand{\ExtensionTok}[1]{{#1}}
\newcommand{\PreprocessorTok}[1]{\textcolor[rgb]{0.56,0.35,0.01}{\textit{{#1}}}}
\newcommand{\AttributeTok}[1]{\textcolor[rgb]{0.77,0.63,0.00}{{#1}}}
\newcommand{\RegionMarkerTok}[1]{{#1}}
\newcommand{\InformationTok}[1]{\textcolor[rgb]{0.56,0.35,0.01}{\textbf{\textit{{#1}}}}}
\newcommand{\WarningTok}[1]{\textcolor[rgb]{0.56,0.35,0.01}{\textbf{\textit{{#1}}}}}
\newcommand{\AlertTok}[1]{\textcolor[rgb]{0.94,0.16,0.16}{{#1}}}
\newcommand{\ErrorTok}[1]{\textcolor[rgb]{0.64,0.00,0.00}{\textbf{{#1}}}}
\newcommand{\NormalTok}[1]{{#1}}
\usepackage{graphicx,grffile}
\makeatletter
\def\maxwidth{\ifdim\Gin@nat@width>\linewidth\linewidth\else\Gin@nat@width\fi}
\def\maxheight{\ifdim\Gin@nat@height>\textheight\textheight\else\Gin@nat@height\fi}
\makeatother
% Scale images if necessary, so that they will not overflow the page
% margins by default, and it is still possible to overwrite the defaults
% using explicit options in \includegraphics[width, height, ...]{}
\setkeys{Gin}{width=\maxwidth,height=\maxheight,keepaspectratio}
\IfFileExists{parskip.sty}{%
\usepackage{parskip}
}{% else
\setlength{\parindent}{0pt}
\setlength{\parskip}{6pt plus 2pt minus 1pt}
}
\setlength{\emergencystretch}{3em}  % prevent overfull lines
\providecommand{\tightlist}{%
  \setlength{\itemsep}{0pt}\setlength{\parskip}{0pt}}
\setcounter{secnumdepth}{0}
% Redefines (sub)paragraphs to behave more like sections
\ifx\paragraph\undefined\else
\let\oldparagraph\paragraph
\renewcommand{\paragraph}[1]{\oldparagraph{#1}\mbox{}}
\fi
\ifx\subparagraph\undefined\else
\let\oldsubparagraph\subparagraph
\renewcommand{\subparagraph}[1]{\oldsubparagraph{#1}\mbox{}}
\fi

%%% Use protect on footnotes to avoid problems with footnotes in titles
\let\rmarkdownfootnote\footnote%
\def\footnote{\protect\rmarkdownfootnote}

%%% Change title format to be more compact
\usepackage{titling}

% Create subtitle command for use in maketitle
\providecommand{\subtitle}[1]{
  \posttitle{
    \begin{center}\large#1\end{center}
    }
}

\setlength{\droptitle}{-2em}

  \title{Heatmap Lipid Cytokines}
    \pretitle{\vspace{\droptitle}\centering\huge}
  \posttitle{\par}
    \author{Katie}
    \preauthor{\centering\large\emph}
  \postauthor{\par}
      \predate{\centering\large\emph}
  \postdate{\par}
    \date{September 12, 2019}


\begin{document}
\maketitle

\subsubsection{import packages}\label{import-packages}

\begin{Shaded}
\begin{Highlighting}[]
\KeywordTok{library}\NormalTok{(tidyverse)}
\KeywordTok{library}\NormalTok{(pheatmap)}
\KeywordTok{library}\NormalTok{(readr)}
\end{Highlighting}
\end{Shaded}

\subsubsection{read in data}\label{read-in-data}

\begin{Shaded}
\begin{Highlighting}[]
\CommentTok{# reading in cytokine data}
\NormalTok{cytokine18 <-}\StringTok{ }\KeywordTok{read_csv}\NormalTok{(}\StringTok{"cytokines0619.csv"}\NormalTok{, }\DataTypeTok{na =} \StringTok{"N/A"}\NormalTok{)}

\CommentTok{# reading in lipid data}
\NormalTok{lipid18 <-}\StringTok{ }\KeywordTok{read_csv}\NormalTok{(}\StringTok{"lipids0619.csv"}\NormalTok{, }\DataTypeTok{na =} \StringTok{'N/A'}\NormalTok{)}

\CommentTok{#reading in clinical data}
\NormalTok{clinic18 <-}\StringTok{ }\KeywordTok{read_csv}\NormalTok{(}\StringTok{"clinical819_rev.csv"}\NormalTok{)}
\end{Highlighting}
\end{Shaded}

\subsubsection{merge lipids and
cytokines}\label{merge-lipids-and-cytokines}

\begin{Shaded}
\begin{Highlighting}[]
\NormalTok{alldata <-}\StringTok{ }\KeywordTok{inner_join}\NormalTok{(cytokine18, lipid18, }\DataTypeTok{by =} \KeywordTok{c}\NormalTok{(}\StringTok{"SampleID"}\NormalTok{, }\StringTok{"LGLLType"}\NormalTok{,  }\StringTok{"Age"}\NormalTok{, }\StringTok{"Gender"}\NormalTok{, }\StringTok{"Immuno"}\NormalTok{, }\StringTok{"STAT3mut"}\NormalTok{, }\StringTok{"STAT3status"}\NormalTok{, }\StringTok{"Statin"}\NormalTok{, }\StringTok{"FishOil"}\NormalTok{))}
\end{Highlighting}
\end{Shaded}

\subsubsection{Merge all data with clinical
data}\label{merge-all-data-with-clinical-data}

\begin{Shaded}
\begin{Highlighting}[]
\NormalTok{alldata <-}\StringTok{ }\KeywordTok{left_join}\NormalTok{(alldata, clinic18, }\DataTypeTok{by =} \KeywordTok{c}\NormalTok{(}\StringTok{"SampleID"}\NormalTok{,  }\StringTok{"Gender"}\NormalTok{))}
\end{Highlighting}
\end{Shaded}

We decided to impute with the median for all of these variables because
distributions can be skewed for LGL patients

\begin{Shaded}
\begin{Highlighting}[]
\NormalTok{alldata %>%}
\StringTok{  }\CommentTok{#filter(Immuno == "N") %>%}
\StringTok{  }\KeywordTok{group_by}\NormalTok{(LGLLType) %>%}
\StringTok{  }\KeywordTok{summarise}\NormalTok{(}\DataTypeTok{med_sICAM =} \KeywordTok{median}\NormalTok{(sICAM_1, }\DataTypeTok{na.rm =} \OtherTok{TRUE}\NormalTok{), }
            \DataTypeTok{med_sFASlig =} \KeywordTok{median}\NormalTok{(sFas_Ligand, }\DataTypeTok{na.rm =} \OtherTok{TRUE}\NormalTok{), }
            \DataTypeTok{med_sFAS =} \KeywordTok{median}\NormalTok{(sFas, }\DataTypeTok{na.rm =} \OtherTok{TRUE}\NormalTok{), }
            \DataTypeTok{med_sVCAM1 =} \KeywordTok{median}\NormalTok{(sVCAM_1, }\DataTypeTok{na.rm =} \OtherTok{TRUE}\NormalTok{))}
\end{Highlighting}
\end{Shaded}

\begin{verbatim}
## # A tibble: 3 x 5
##   LGLLType     med_sICAM med_sFASlig med_sFAS med_sVCAM1
##   <chr>            <dbl>       <dbl>    <dbl>      <dbl>
## 1 NK-LGLL         196501        165.    8198.   1322002 
## 2 Normal Donor    150931         75     6192     777563 
## 3 T-LGLL          207846        115.    7522    1185792.
\end{verbatim}

Now impute using the above medians for normal patients. These values do
not change with / without Immuno line because patients taking
immunosuppression drugs were leukemic patients rather than normals

\begin{Shaded}
\begin{Highlighting}[]
\NormalTok{alldata$sICAM_1[}\DecValTok{71}\NormalTok{] <-}\StringTok{ }\DecValTok{150931} \CommentTok{#patient 73 is at row 71}
\NormalTok{alldata$sFas_Ligand[}\DecValTok{71}\NormalTok{]<-}\StringTok{ }\DecValTok{75}
\NormalTok{alldata$sFas[}\DecValTok{71}\NormalTok{]<-}\DecValTok{6192}
\NormalTok{alldata$sVCAM_1[}\DecValTok{71}\NormalTok{]<-}\StringTok{ }\DecValTok{777563}
\end{Highlighting}
\end{Shaded}

Check for other missing values on biomarkers

\begin{Shaded}
\begin{Highlighting}[]
\NormalTok{alldata %>%}
\StringTok{  }\NormalTok{dplyr::}\KeywordTok{select}\NormalTok{(MIG:SMC26_1) %>%}
\StringTok{  }\KeywordTok{is.na}\NormalTok{() %>%}
\StringTok{  }\KeywordTok{sum}\NormalTok{()}
\end{Highlighting}
\end{Shaded}

\begin{verbatim}
## [1] 0
\end{verbatim}

No other missing values

\subsubsection{Remove Patients on immunosuppressive
treatment}\label{remove-patients-on-immunosuppressive-treatment}

\textbf{ALSO remove patient ID == 20} Patient 20 was excluded because
they were initially incorrectly identified and were found to be on
immunosuppressive treatment

\begin{Shaded}
\begin{Highlighting}[]
\NormalTok{alldata_noImmuno <-}\StringTok{ }\NormalTok{alldata %>%}
\StringTok{  }\KeywordTok{filter}\NormalTok{(Immuno ==}\StringTok{ "N"}\NormalTok{) %>%}
\StringTok{  }\KeywordTok{filter}\NormalTok{(SampleID !=}\StringTok{ }\DecValTok{20}\NormalTok{)}
\KeywordTok{rm}\NormalTok{(alldata)}
\end{Highlighting}
\end{Shaded}

\section{Pre-processing for Heatmap}\label{pre-processing-for-heatmap}

\subsubsection{Create biomarker
dataframe}\label{create-biomarker-dataframe}

selecting the biomarkers from alldata\_noImmuno

\begin{Shaded}
\begin{Highlighting}[]
\NormalTok{biomarkers_noImmuno <-}\StringTok{ }\NormalTok{alldata_noImmuno %>%}
\StringTok{  }\NormalTok{dplyr::}\KeywordTok{select}\NormalTok{(MIG:SMC26_1)}

\CommentTok{# drop tgfb3 because it has 0 variance (all observations below limit of detection)}
\NormalTok{biomarkers_noImmuno <-}\StringTok{ }\NormalTok{biomarkers_noImmuno %>%}
\StringTok{  }\NormalTok{dplyr::}\KeywordTok{select}\NormalTok{(-TGFB3)}
\end{Highlighting}
\end{Shaded}

\subsubsection{log10 transform of data}\label{log10-transform-of-data}

\begin{Shaded}
\begin{Highlighting}[]
\NormalTok{biomarkers_noImmuno <-}\StringTok{ }\KeywordTok{log10}\NormalTok{(biomarkers_noImmuno)}
\end{Highlighting}
\end{Shaded}

\subsubsection{scale data}\label{scale-data}

\begin{Shaded}
\begin{Highlighting}[]
\NormalTok{biomarkers_noImmuno <-}\StringTok{ }\KeywordTok{scale}\NormalTok{(biomarkers_noImmuno)}
\end{Highlighting}
\end{Shaded}

\subsubsection{set row names for main matrix
values}\label{set-row-names-for-main-matrix-values}

\begin{Shaded}
\begin{Highlighting}[]
\KeywordTok{rownames}\NormalTok{(biomarkers_noImmuno) <-}\StringTok{ }\NormalTok{alldata_noImmuno$SampleID}

\KeywordTok{write.csv}\NormalTok{(biomarkers_noImmuno, }\DataTypeTok{file =} \StringTok{"biomarkers_noImmuno.csv"}\NormalTok{)}
\end{Highlighting}
\end{Shaded}

\subsubsection{make col/row annotation
dataframes}\label{make-colrow-annotation-dataframes}

Because we want to annotate the heatmap with these variables, we're
making a dataframe of them. \textbf{Ensure that the annotation dataframe
and the biomarker dataframe have the same patient order}

\begin{Shaded}
\begin{Highlighting}[]
\NormalTok{annrow_df <-}\StringTok{ }\NormalTok{alldata_noImmuno %>%}
\StringTok{  }\KeywordTok{select}\NormalTok{(Gender, LGLLType, STAT3status, MCV, Hgb, ANC)}
\KeywordTok{rownames}\NormalTok{(annrow_df) <-}\StringTok{ }\NormalTok{alldata_noImmuno$SampleID}
\end{Highlighting}
\end{Shaded}

\begin{verbatim}
## Warning: Setting row names on a tibble is deprecated.
\end{verbatim}

\begin{Shaded}
\begin{Highlighting}[]
\NormalTok{annrow_df$Gender <-}\StringTok{ }\KeywordTok{factor}\NormalTok{(annrow_df$Gender)}
\NormalTok{annrow_df$LGLLType <-}\StringTok{ }\KeywordTok{factor}\NormalTok{(annrow_df$LGLLType)}
\NormalTok{annrow_df$STAT3status <-}\StringTok{ }\KeywordTok{factor}\NormalTok{(annrow_df$STAT3status)}

\KeywordTok{write.csv}\NormalTok{(annrow_df, }\DataTypeTok{file =} \StringTok{"annotations.csv"}\NormalTok{)}
\end{Highlighting}
\end{Shaded}

\section{Create the Heatmap}\label{create-the-heatmap}

\begin{Shaded}
\begin{Highlighting}[]
\CommentTok{#loads in the files for plotting and set sampleID (currently column 1) as the rownames}
\NormalTok{biomarkers_noImmuno <-}\StringTok{ }\KeywordTok{read.csv}\NormalTok{(}\StringTok{"biomarkers_noImmuno.csv"}\NormalTok{, }\DataTypeTok{row.names =} \DecValTok{1}\NormalTok{)}
\NormalTok{sample_ann_df <-}\StringTok{ }\KeywordTok{read.csv}\NormalTok{(}\StringTok{"annotations.csv"}\NormalTok{, }\DataTypeTok{row.names =} \DecValTok{1}\NormalTok{)}
\end{Highlighting}
\end{Shaded}

\begin{Shaded}
\begin{Highlighting}[]
\CommentTok{#sets annotation colors}
\NormalTok{ann_colors =}\StringTok{ }\KeywordTok{list}\NormalTok{(}\DataTypeTok{Gender =} \KeywordTok{c}\NormalTok{(}\DataTypeTok{M =} \StringTok{'turquoise'}\NormalTok{, }\DataTypeTok{F =} \StringTok{'plum'}\NormalTok{), }
                  \DataTypeTok{LGLLType =} \KeywordTok{c}\NormalTok{(}\StringTok{"NK-LGLL"} \NormalTok{=}\StringTok{ "gold"}\NormalTok{, }\StringTok{"Normal Donor"} \NormalTok{=}\StringTok{ 'orchid1'}\NormalTok{, }\StringTok{"T-LGLL"} \NormalTok{=}\StringTok{ "deepskyblue"}\NormalTok{), }
                  \DataTypeTok{STAT3status =} \KeywordTok{c}\NormalTok{(}\DataTypeTok{D661I =} \StringTok{'deepskyblue'}\NormalTok{, }\DataTypeTok{D661Y =} \StringTok{'palegreen'}\NormalTok{, }\DataTypeTok{K658R =} \StringTok{'pink'}\NormalTok{, }\DataTypeTok{N647I =} \StringTok{'darkorange'}\NormalTok{, }\DataTypeTok{S614R =} \StringTok{'palegoldenrod'}\NormalTok{, }\DataTypeTok{WT =} \StringTok{'azure2'}\NormalTok{, }\DataTypeTok{Y640F =} \StringTok{'darkorchid'}\NormalTok{, }\StringTok{'Y640F, D661Y'} \NormalTok{=}\StringTok{ 'royalblue'}\NormalTok{, }\StringTok{'Y640F, I659L'} \NormalTok{=}\StringTok{ 'violet'}\NormalTok{, }\StringTok{'Y640F, Q643H'} \NormalTok{=}\StringTok{ 'black'}\NormalTok{))}

\CommentTok{#plots heatmap}
\KeywordTok{options}\NormalTok{(}\StringTok{'repr.plot.height'} \NormalTok{=}\StringTok{ }\DecValTok{12}\NormalTok{, }\StringTok{'repr.plot.width'} \NormalTok{=}\StringTok{ }\DecValTok{12}\NormalTok{)}
\NormalTok{p <-}\StringTok{ }\KeywordTok{pheatmap}\NormalTok{(}\KeywordTok{t}\NormalTok{(biomarkers_noImmuno),}
              \DataTypeTok{color =} \KeywordTok{colorRampPalette}\NormalTok{(}\KeywordTok{c}\NormalTok{(}\StringTok{"blue"}\NormalTok{, }\StringTok{"white"}\NormalTok{, }\StringTok{"red"}\NormalTok{))(}\DataTypeTok{n =} \DecValTok{1000}\NormalTok{),}
              \DataTypeTok{clustering_distance_rows =} \StringTok{"euclidean"}\NormalTok{,}
              \DataTypeTok{clustering_distance_cols =} \StringTok{'euclidean'}\NormalTok{,}
              \DataTypeTok{annotation_col =} \NormalTok{sample_ann_df,}
              \DataTypeTok{annotation_colors =} \NormalTok{ann_colors,}
              \DataTypeTok{main =} \StringTok{'Lipid and Cytokine Clustered Heatmap'}\NormalTok{, }
              \DataTypeTok{border_color =} \OtherTok{NA}\NormalTok{,}
              \DataTypeTok{show_colnames =} \OtherTok{FALSE}\NormalTok{,}
              \DataTypeTok{height =} \DecValTok{9}\NormalTok{,}
              \DataTypeTok{width =} \FloatTok{9.5}\NormalTok{,}
              \DataTypeTok{filename =} \StringTok{"newheatmap.png"}\NormalTok{)}
\end{Highlighting}
\end{Shaded}


\end{document}
